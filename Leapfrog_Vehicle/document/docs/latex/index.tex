This is the G\+IT push of the project software for Leapfrog as of 22nd August 2024. It is hosted on a private repostiroy belonging to Howard Hall and will be regularly synced with the S\+E\+RC gitlab.

This repository will contain a version control of the associated software and branch from the main. Any long term changes will then be merged back with main after review.

Currently the main contains two main sections\+: flight software for the S\+T\+M32 and the ground station software, in the associated folders.

By our own convention, version 0.\+0 will denote the files uploaded as of 22nd August 2024.

To see Documentation, open documents/docs/html/index.\+html on local computer since github doesn\textquotesingle{}t allow so many files uploaded {\ttfamily \+:(}

Please adhere to at least the basic practises of git push/pull/commits/merges. See instructions below.


\begin{DoxyItemize}
\item \href{\#developer-workflow}{\texttt{ Developer Workflow}}
\begin{DoxyItemize}
\item \href{\#create-a-new-branch}{\texttt{ 1. Create a New Branch}}
\item \href{\#make-changes-and-commit}{\texttt{ 2. Make Changes and Commit}}
\item \href{\#create-a-pull-request-pr}{\texttt{ 3. Create a Pull Request (PR)}}
\item \href{\#review-and-approval}{\texttt{ 4. Review and Approval}}
\item \href{\#merging-the-pr-into-develop}{\texttt{ 5. Merging the PR into {\ttfamily develop}}}
\item \href{\#clean-up}{\texttt{ 6. Clean Up}}
\end{DoxyItemize}
\item \href{\#how-to-git}{\texttt{ First Timers}}
\begin{DoxyItemize}
\item \href{\#prerequisites}{\texttt{ Prerequisites}}
\item \href{\#setting-up-your-ssh-key}{\texttt{ Setting Up Your S\+SH Key}}
\item \href{\#cloning-the-repository}{\texttt{ Cloning the Repository}}
\end{DoxyItemize}
\end{DoxyItemize}





temporary todo task \tabulinesep=1mm
\begin{longtabu}spread 0pt [c]{*{5}{|X[-1]}|}
\hline
\PBS\centering \cellcolor{\tableheadbgcolor}\textbf{ Task  }&\PBS\centering \cellcolor{\tableheadbgcolor}\textbf{ Status  }&\PBS\centering \cellcolor{\tableheadbgcolor}\textbf{ Assignee  }&\PBS\centering \cellcolor{\tableheadbgcolor}\textbf{ Due Date  }&\PBS\centering \cellcolor{\tableheadbgcolor}\textbf{ Priority   }\\\cline{1-5}
\endfirsthead
\hline
\endfoot
\hline
\PBS\centering \cellcolor{\tableheadbgcolor}\textbf{ Task  }&\PBS\centering \cellcolor{\tableheadbgcolor}\textbf{ Status  }&\PBS\centering \cellcolor{\tableheadbgcolor}\textbf{ Assignee  }&\PBS\centering \cellcolor{\tableheadbgcolor}\textbf{ Due Date  }&\PBS\centering \cellcolor{\tableheadbgcolor}\textbf{ Priority   }\\\cline{1-5}
\endhead
Jenkins/github action for doxygen  &Not Started  &Charles  &2024-\/10-\/20  &Low   \\\cline{1-5}
Improve Documentation(doxygen)  &Not Started  &Any Team Member  &2024-\/10-\/20  &Mid   \\\cline{1-5}
\end{longtabu}


no touch main!

Whenever a new feature, bug fix, or change needs to be made, start by creating a new branch. This ensures that the {\ttfamily develop} branch remains stable and can receive tested code only.


\begin{DoxyEnumerate}
\item Fetch the latest updates from the remote repository\+: \`{}\`{}\`{}bash git fetch origin \`{}\`{}\`{}
\item Switch to the {\ttfamily develop} branch to ensure you\textquotesingle{}re branching off the latest state\+: \`{}\`{}\`{}bash git checkout develop \`{}\`{}\`{}
\item Pull the latest changes to avoid working on an outdated version\+: \`{}\`{}\`{}bash git pull origin develop \`{}\`{}\`{}
\item Create and switch to a new branch for your changes\+: \`{}\`{}\`{}bash git checkout -\/b feature/your-\/feature-\/name \`{}\`{}{\ttfamily 
\begin{DoxyItemize}
\item $\ast$\+Tip\+:$\ast$ Use descriptive branch names prefixed with the type of change, such asfeature/{\ttfamily ,}fix/{\ttfamily , or}hotfix/\`{}.
\end{DoxyItemize}}
\end{DoxyEnumerate}

{\ttfamily }

{\ttfamily  After switching to your new branch, you can begin making changes.}

{\ttfamily }

{\ttfamily 
\begin{DoxyEnumerate}
\item Add the changes to be staged for commit\+: \`{}\`{}\`{}bash git add . \`{}\`{}\`{}
\item Commit the changes with a descriptive message\+: \`{}\`{}\`{}bash git commit -\/m \char`\"{}\+Add detailed description of the changes made\char`\"{} \`{}\`{}\`{}
\item Push your branch to the remote repository\+: \`{}\`{}\`{}bash git push origin feature/your-\/feature-\/name \`{}\`{}\`{}
\end{DoxyEnumerate}}

{\ttfamily }

{\ttfamily  Once changes are made and pushed to the new branch, create a Pull Request (PR) to merge it into the {\ttfamily develop} branch. This process allows for code review and testing before merging.}

{\ttfamily 
\begin{DoxyEnumerate}
\item Go to the Git\+Hub repository.
\item Navigate to the \char`\"{}\+Pull Requests\char`\"{} tab.
\item Select \char`\"{}\+New Pull Request.\char`\"{}
\item Set the base branch to {\ttfamily develop} and compare it with your feature branch.
\item Add a title and description to explain the changes.
\item Submit the PR for review.
\end{DoxyEnumerate}}

{\ttfamily }

{\ttfamily 
\begin{DoxyItemize}
\item Once the PR is submitted, the code will be reviewed by peers or maintainers.
\item Address any requested changes or feedback.
\item Once approved, the PR can be merged into the {\ttfamily develop} branch.
\end{DoxyItemize}}

{\ttfamily }

{\ttfamily  }

{\ttfamily 
\begin{DoxyEnumerate}
\item Once the PR is approved, ensure your {\ttfamily develop} branch is up to date\+: \`{}\`{}\`{}bash git checkout develop git pull origin develop \`{}\`{}\`{}
\item Merge the feature branch into {\ttfamily develop}\+: \`{}\`{}\`{}bash git merge --no-\/ff feature/your-\/feature-\/name \`{}\`{}{\ttfamily 
\begin{DoxyItemize}
\item $\ast$\+Tip\+:$\ast$ The--no-\/ff\`{} flag ensures that the merge creates a commit, preserving the history of the branch.
\end{DoxyItemize}}
\item {\ttfamily Push the updated {\ttfamily develop} branch to the remote repository\+: \`{}\`{}\`{}bash git push origin develop \`{}\`{}\`{}}
\end{DoxyEnumerate}}

{\ttfamily {\ttfamily }}

{\ttfamily {\ttfamily  Once the feature is merged, you can delete the branch locally and remotely\+:}}

{\ttfamily {\ttfamily }}

{\ttfamily {\ttfamily 
\begin{DoxyEnumerate}
\item Delete the local branch\+: \`{}\`{}\`{}bash git branch -\/d feature/your-\/feature-\/name \`{}\`{}\`{}
\item Delete the remote branch\+: \`{}\`{}\`{}bash git push origin --delete feature/your-\/feature-\/name \`{}\`{}\`{}
\end{DoxyEnumerate}}}

{\ttfamily {\ttfamily \href{\#table-of-contents}{\texttt{ Back to Top}}}}

{\ttfamily {\ttfamily 

}}

{\ttfamily {\ttfamily }}

{\ttfamily {\ttfamily }}

{\ttfamily {\ttfamily }}

{\ttfamily {\ttfamily Before you begin, ensure you have the following installed on your machine\+:
\begin{DoxyItemize}
\item Git
\item A text editor (e.\+g., {\bfseries{Visual Studio Code}} recommended, Atom, etc.)
\end{DoxyItemize}}}

{\ttfamily {\ttfamily }}

{\ttfamily {\ttfamily Only first time. Follow these steps to create and add your S\+SH key\+:}}

{\ttfamily {\ttfamily 
\begin{DoxyEnumerate}
\item {\bfseries{Open your terminal (or Git Bash on Windows)}}.
\item {\bfseries{Generate a new S\+SH key}}\+:
\begin{DoxyItemize}
\item Use the command {\ttfamily ssh-\/keygen -\/t rsa -\/b 4096 -\/C \char`\"{}your\+\_\+email@example.\+com\char`\"{}}, replacing {\ttfamily your\+\_\+email@example.\+com} with the email address associated with your Git\+Hub account.
\end{DoxyItemize}
\item {\bfseries{When prompted, press Enter}} to accept the default file location. If you want to add a passphrase for additional security, you can do that as well.
\item {\bfseries{Start the S\+SH agent}}\+:
\begin{DoxyItemize}
\item Run the command {\ttfamily eval \char`\"{}\$(ssh-\/agent -\/s)\char`\"{}}.
\end{DoxyItemize}
\item {\bfseries{Add your S\+SH private key to the S\+SH agent}}\+:
\begin{DoxyItemize}
\item Use {\ttfamily ssh-\/add $\sim$/.ssh/id\+\_\+rsa}.
\end{DoxyItemize}
\item {\bfseries{Copy your S\+SH key to the clipboard}}\+:
\begin{DoxyItemize}
\item For Windows, use {\ttfamily clip $<$ $\sim$/.ssh/id\+\_\+rsa.\+pub}.
\item For mac\+OS, use {\ttfamily pbcopy $<$ $\sim$/.ssh/id\+\_\+rsa.\+pub}.
\item For Linux, use {\ttfamily xclip -\/sel clip $<$ $\sim$/.ssh/id\+\_\+rsa.\+pub}.
\end{DoxyItemize}
\item {\bfseries{Add the S\+SH key to your Git\+Hub account}}\+:
\begin{DoxyItemize}
\item Go to your Git\+Hub account.
\item Navigate to {\bfseries{Settings}} $>$ {\bfseries{S\+SH and G\+PG keys}} $>$ {\bfseries{New S\+SH key}}.
\item Paste your S\+SH key and give it a title.
\item Click {\bfseries{Add S\+SH key}}.
\end{DoxyItemize}
\end{DoxyEnumerate}}}

{\ttfamily {\ttfamily }}

{\ttfamily {\ttfamily Once your S\+SH key is set up, you can clone the repository using the command {\ttfamily git clone git@github.\+com\+:hall-\/engine/leapfrog.\+git}.}}

{\ttfamily {\ttfamily Navigate into the project directory with the command {\ttfamily cd leapfrog}.}}

{\ttfamily {\ttfamily See \href{\#developer-workflow}{\texttt{ Developer Workflow}} for implementing changes}}

{\ttfamily {\ttfamily \href{\#table-of-contents}{\texttt{ Back to Top}}}}

{\ttfamily {\ttfamily }}

{\ttfamily {\ttfamily {\bfseries{Will Christian}} \href{mailto:william.christian@usc.edu}{\texttt{ william.\+christian@usc.\+edu}}}}

{\ttfamily {\ttfamily {\bfseries{Varick John}} \href{mailto:vjohn@usc.edu}{\texttt{ vjohn@usc.\+edu}}}}

{\ttfamily {\ttfamily {\bfseries{Howard Hall}} \href{mailto:hahall@usc.edu}{\texttt{ hahall@usc.\+edu}}}}

{\ttfamily {\ttfamily {\bfseries{Briana Zeggane}} \href{mailto:zeggane@usc.edu}{\texttt{ zeggane@usc.\+edu}}}}

{\ttfamily {\ttfamily {\bfseries{Trevor Gross}} \href{mailto:trevorg@usc.edu}{\texttt{ trevorg@usc.\+edu}} }}