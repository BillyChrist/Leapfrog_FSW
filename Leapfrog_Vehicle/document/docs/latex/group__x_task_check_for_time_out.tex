\hypertarget{group__x_task_check_for_time_out}{}\section{x\+Task\+Check\+For\+Time\+Out}
\label{group__x_task_check_for_time_out}\index{xTaskCheckForTimeOut@{xTaskCheckForTimeOut}}
\mbox{\hyperlink{task_8h_source}{task.\+h}} 
\begin{DoxyPre}BaseType\_t xTaskCheckForTimeOut( TimeOut\_t * const pxTimeOut, TickType\_t * const pxTicksToWait );\end{DoxyPre}


Determines if px\+Ticks\+To\+Wait ticks has passed since a time was captured using a call to v\+Task\+Set\+Time\+Out\+State(). The captured time includes the tick count and the number of times the tick count has overflowed.


\begin{DoxyParams}{Parameters}
{\em px\+Time\+Out} & The time status as captured previously using v\+Task\+Set\+Time\+Out\+State. If the timeout has not yet occurred, it is updated to reflect the current time status. \\
\hline
{\em px\+Ticks\+To\+Wait} & The number of ticks to check for timeout i.\+e. if px\+Ticks\+To\+Wait ticks have passed since px\+Time\+Out was last updated (either by v\+Task\+Set\+Time\+Out\+State() or x\+Task\+Check\+For\+Time\+Out()), the timeout has occurred. If the timeout has not occurred, px\+T\+Icks\+To\+Wait is updated to reflect the number of remaining ticks.\\
\hline
\end{DoxyParams}
\begin{DoxyReturn}{Returns}
If timeout has occurred, pd\+T\+R\+UE is returned. Otherwise pd\+F\+A\+L\+SE is returned and px\+Ticks\+To\+Wait is updated to reflect the number of remaining ticks.
\end{DoxyReturn}
\begin{DoxySeeAlso}{See also}
\href{https://www.freertos.org/xTaskCheckForTimeOut.html}{\texttt{ https\+://www.\+freertos.\+org/x\+Task\+Check\+For\+Time\+Out.\+html}}
\end{DoxySeeAlso}
Example Usage\+: 
\begin{DoxyPre}
   // Driver library function used to receive uxWantedBytes from an Rx buffer
   // that is filled by a UART interrupt. If there are not enough bytes in the
   // Rx buffer then the task enters the Blocked state until it is notified that
   // more data has been placed into the buffer. If there is still not enough
   // data then the task re-enters the Blocked state, and xTaskCheckForTimeOut()
   // is used to re-calculate the Block time to ensure the total amount of time
   // spent in the Blocked state does not exceed MAX\_TIME\_TO\_WAIT. This
   // continues until either the buffer contains at least uxWantedBytes bytes,
   // or the total amount of time spent in the Blocked state reaches
   // MAX\_TIME\_TO\_WAIT – at which point the task reads however many bytes are
   // available up to a maximum of uxWantedBytes.\end{DoxyPre}



\begin{DoxyPre}   size\_t xUART\_Receive( uint8\_t *pucBuffer, size\_t uxWantedBytes )
   \{
   size\_t uxReceived = 0;
   TickType\_t xTicksToWait = MAX\_TIME\_TO\_WAIT;
   TimeOut\_t xTimeOut;\end{DoxyPre}



\begin{DoxyPre}    // Initialize xTimeOut.  This records the time at which this function
    // was entered.
    vTaskSetTimeOutState( \&xTimeOut );\end{DoxyPre}



\begin{DoxyPre}    // Loop until the buffer contains the wanted number of bytes, or a
    // timeout occurs.
    while( UART\_bytes\_in\_rx\_buffer( pxUARTInstance ) < uxWantedBytes )
    \{
        // The buffer didn't contain enough data so this task is going to
        // enter the Blocked state. Adjusting xTicksToWait to account for
        // any time that has been spent in the Blocked state within this
        // function so far to ensure the total amount of time spent in the
        // Blocked state does not exceed MAX\_TIME\_TO\_WAIT.
        if( xTaskCheckForTimeOut( \&xTimeOut, \&xTicksToWait ) != pdFALSE )
        \{
            //Timed out before the wanted number of bytes were available,
            // exit the loop.
            break;
        \}\end{DoxyPre}



\begin{DoxyPre}        // Wait for a maximum of xTicksToWait ticks to be notified that the
        // receive interrupt has placed more data into the buffer.
        ulTaskNotifyTake( pdTRUE, xTicksToWait );
    \}\end{DoxyPre}



\begin{DoxyPre}    // Attempt to read uxWantedBytes from the receive buffer into pucBuffer.
    // The actual number of bytes read (which might be less than
    // uxWantedBytes) is returned.
    uxReceived = UART\_read\_from\_receive\_buffer( pxUARTInstance,
                                                pucBuffer,
                                                uxWantedBytes );\end{DoxyPre}



\begin{DoxyPre}    return uxReceived;
   \}
\end{DoxyPre}
 