\hypertarget{group__x_stream_buffer_set_trigger_level}{}\section{x\+Stream\+Buffer\+Set\+Trigger\+Level}
\label{group__x_stream_buffer_set_trigger_level}\index{xStreamBufferSetTriggerLevel@{xStreamBufferSetTriggerLevel}}
\mbox{\hyperlink{stream__buffer_8h_source}{stream\+\_\+buffer.\+h}}


\begin{DoxyPre}
BaseType\_t xStreamBufferSetTriggerLevel( StreamBufferHandle\_t xStreamBuffer, size\_t xTriggerLevel );
\end{DoxyPre}


A stream buffer\textquotesingle{}s trigger level is the number of bytes that must be in the stream buffer before a task that is blocked on the stream buffer to wait for data is moved out of the blocked state. For example, if a task is blocked on a read of an empty stream buffer that has a trigger level of 1 then the task will be unblocked when a single byte is written to the buffer or the task\textquotesingle{}s block time expires. As another example, if a task is blocked on a read of an empty stream buffer that has a trigger level of 10 then the task will not be unblocked until the stream buffer contains at least 10 bytes or the task\textquotesingle{}s block time expires. If a reading task\textquotesingle{}s block time expires before the trigger level is reached then the task will still receive however many bytes are actually available. Setting a trigger level of 0 will result in a trigger level of 1 being used. It is not valid to specify a trigger level that is greater than the buffer size.

A trigger level is set when the stream buffer is created, and can be modified using x\+Stream\+Buffer\+Set\+Trigger\+Level().


\begin{DoxyParams}{Parameters}
{\em x\+Stream\+Buffer} & The handle of the stream buffer being updated.\\
\hline
{\em x\+Trigger\+Level} & The new trigger level for the stream buffer.\\
\hline
\end{DoxyParams}
\begin{DoxyReturn}{Returns}
If x\+Trigger\+Level was less than or equal to the stream buffer\textquotesingle{}s length then the trigger level will be updated and pd\+T\+R\+UE is returned. Otherwise pd\+F\+A\+L\+SE is returned. 
\end{DoxyReturn}
